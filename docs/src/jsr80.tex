
%%%%%%%%%%%%%%%%%%%%%%%%%%%%%%%%%%%%%%%%
% Start of Document
%

\documentclass{article}
\usepackage{fancyhdr}
\usepackage{graphics}
\usepackage{latexsym}

\addtolength{\textwidth}{+2in}
\addtolength{\oddsidemargin}{-1in}
\addtolength{\evensidemargin}{-1in}

\begin{document}
\pagestyle{empty}

\newcommand{\mytitle}[0]{JSR80 API Specification}
\newcommand{\myauthor}[0]{Dan Streetman}

%%%%%%%%%%%%%%%%%%%%%%%%%%%%%%%%%%%%%%%%
% Titlepage
%

\begin{titlepage}

\begin{center}
\Huge{\textbf{\mytitle}}
\end{center}

\vfill
\vfill

\begin{flushright}
\large{\myauthor}
\\
\large{ddstreet@us.ibm.com}
\\
\large{\today}
\end{flushright}

\vfill

\end{titlepage}

\newpage


%%%%%%%%%%%%%%%%%%%%%%%%%%%%%%%%%%%%%%%
% Page setup
%

\pagestyle{fancy}
\fancyhf{}
\renewcommand{\sectionmark}[1]{\markright{\emph{\mytitle}}}
\renewcommand{\subsectionmark}[1]{\markright{\emph{\mytitle}}}
\renewcommand{\subsubsectionmark}[1]{\markright{\emph{\mytitle}}}
\fancyhead[LE,RO]{\bfseries\thepage}
\fancyhead[LO]{\rightmark}


%%%%%%%%%%%%%%%%%%%%%%%%%%%%%%%%%%%%%%%
% Table of Contents
%

\tableofcontents

\listoftables

\listoffigures

\newpage

%%%%%%%%%%%%%%%%%%%%%%%%%%%%%%%%%%%%%%%%
% Entry Point
%

\section{Entry Point}

The entry point for javax.usb is the \emph{UsbHostManager} class.  This class instantiates
and provides access to the implementation's main access class, which is an implementation of
\emph{UsbServices}.  This implementation class must provide a no-parameter constructor
and it must be listed in the (required) properties file, \emph{javax/usb/res/jusb.properties},
as the value for the key \emph{javax.usb.os.UsbServices}.

The \emph{UsbServices} provides add and remove methods for \emph{UsbServicesListener}s, which
monitors when devices are connected or added to the system.  Also the class provides access
to the root \emph{UsbHub}.  This hub is the root of the entire USB topology tree.  It does
not correspond to a physical hub, but instead is a virtual hub to which all physical Host Controller
hubs are connected.  Each of the physical Host Controller hubs contains the devices actually
connected to the system.

%%%%%%%%%%%%%%%%%%%%%%%%%%%%%%%%%%%%%%%%
% Topology Tree
%

\section{Topology Tree}

The USB topology tree is a representation of the physical device connection layout.  It
consists of a tree-based structure of \emph{UsbDevice}s, some of which are \emph{UsbHub}s.
It starts at the virtual root \emph{UsbHub}, to which devices (which are actually
Host Controller hubs) are connected.  Each of these hubs may have devices connected on one or
more of its \emph{UsbPort}s.  These connected devices may actually be a hub, in which case they
may have more devices connected to their port(s).  This is the way that USB devices are connected
to a system, with the exception of course of the virtual root hub.  That is created and managed
by the JSR80 implementation.  See the section on \emph{UsbHubs and UsbDevices} for details on
the virtual root hub.

%%%%%%%%%%%%%%%%%%%%%%%%%%%%%%%%%%%%%%%%
% UsbHubs and UsbDevices
%

\section{UsbHubs and UsbDevices}

A \emph{UsbHub} is an extended \emph{UsbDevice}.  It has methods specific to a hub.  Those
methods allow the user to access the hub's \emph{UsbPort}s and attached devices.  

A \emph{UsbDevice} represents a physical USB device.  It provides methods for accessing the
parent port that the device is connected to, accessing the device's descriptor, accessing
the device's configurations and strings, and adding or removing a \emph{UsbDeviceListener}
which monitors when the device is disconnected as well as when communication occurs on the
Default Control Pipe.  It also provides methods to communicate via the Default Control Pipe.

%%%%%%%%%%%%%%%%%%%%%%%%%%%%%%%%%%%%%%%%
% UsbDevice Components
%

\section{UsbDevice Components}

Each \emph{UsbDevice} is made up of several parts.  These parts start with the device's
configuration(s).  A USB device is required to contain at least one configuration, and
only one configuration may be active at once, meaning the device must be configured
in one and only one certain way, that it defines.  A device may also be in an unconfigured
state, but normally the operating system software initializes the device to its default
configuration, which is configuration number 1 (0 is reserved for the unconfigured state).
Each configuration is represented by a \emph{UsbConfig}.  The device maintains which
of its configs is active, if any.

\end{document}