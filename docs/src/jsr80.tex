
%%%%%%%%%%%%%%%%%%%%%%%%%%%%%%%%%%%%%%%%
% Start of Document
%

\documentclass{article}
\usepackage{fancyhdr}
\usepackage{graphicx}
\usepackage{latexsym}
\usepackage{times}

\addtolength{\textwidth}{+1in}
\addtolength{\oddsidemargin}{-0.5in}
\addtolength{\evensidemargin}{-0.5in}

\newcommand{\mytitle}[0]{JSR80 API Specification}
\newcommand{\myauthor}[0]{Dan Streetman}
\newcommand{\mydate}[0]{\today}

\newcommand{\myclass}[1]{\emph{#1}}
\newcommand{\myinterface}[1]{\emph{#1}}
\newcommand{\mypackage}[1]{\tt#1\rm}
\newcommand{\mymethod}[1]{\emph{#1}}
\newcommand{\myfield}[1]{\emph{#1}}

\newcommand{\mysectionend}[0]{\vfill\pagebreak[1]}

\newcommand{\myfigure}[3]{\begin{figure}[htbp]\centering\includegraphics[width=0.75\textwidth]{#1}\caption{#2}\label{#3}\end{figure}}

\begin{document}
\pagestyle{empty}

\pagenumbering{roman}

%%%%%%%%%%%%%%%%%%%%%%%%%%%%%%%%%%%%%%%%
% Titlepage
%

\begin{titlepage}

\begin{center}
\Huge{\textbf{\mytitle}}
\end{center}

\vfill
\vfill

\begin{flushright}
\large{\myauthor}
\\
\large{ddstreet@ieee.org}
\\
\large{\mydate}
\end{flushright}

\vfill

\end{titlepage}

%%%%%%%%%%%%%%%%%%%%%%%%%%%%%%%%%%%%%%%
% Page setup
%

\pagestyle{fancy}
\fancyhf{}
\renewcommand{\sectionmark}[1]{\markright{\emph{\mytitle}}}
\renewcommand{\subsectionmark}[1]{\markright{\emph{\mytitle}}}
\renewcommand{\subsubsectionmark}[1]{\markright{\emph{\mytitle}}}
\fancyhead[L,RO]{\bfseries\thepage}
\fancyhead[LO]{\rightmark}

%%%%%%%%%%%%%%%%%%%%%%%%%%%%%%%%%%%%%%%
% Table of Contents
%

\tableofcontents

\listoftables

\listoffigures

\pagebreak

%%%%%%%%%%%%%%%%%%%%%%%%%%%%%%%%%%%%%%%%
% Preface
%

\section{Preface}

\subsection{Introduction}

JSR80 is the Java Specification Request concerned with communication with
Universal Serial Bus (USB) devices.  This document describes the API associated
with this JSR.  

\mysectionend

\pagenumbering{arabic}

%%%%%%%%%%%%%%%%%%%%%%%%%%%%%%%%%%%%%%%%
% USB Bus Topology
%

\section{USB Bus Topology}

The structure of the USB bus is topological.  The root
of the USB physical bus is a physical Host Controller, which is contained
in a computer system.  Host Controllers are usually on-board or may be
add-in PCI cards.  They appear logically as hubs, so they are the root hub
in their own topology tree.  Connected to them are one or more USB devices,
and since USB hubs are a type of USB device, this topology may continue
down until there are no more connected devices.  The non-hub devices form
the leaves of the topology tree, while the hub-type devices form the branches
of the topology tree.  This forms the physical device topology.

\myfigure{figs/logical_bus_topology}{Logical Bus Topology}{bus_topology}

The javax.usb API closely matches this physical device topology, as shown in
figure \ref{bus_topology}.  However the root of the topology is slightly different.
First, the entry point of javax.usb is the \myclass{UsbHostManager} class, shown at the
top of the tree.  This class instantiates the platform-specific instance of
the \myinterface{UsbServices} interface.  From the \myinterface{UsbServices} instance, the virtual root
\myinterface{UsbHub} is available.  This hub is created by the implementation and does
nothing except manage the actual physical devices.  Each hub connected to
the virtual root hub represents a physical Host Controller hub that is
present in the system.  Connected to those hubs are the real externally-connected
physical devices connected to the system, including external hubs.

\mysectionend

%%%%%%%%%%%%%%%%%%%%%%%%%%%%%%%%%%%%%%%%
% USB Device Hierarchy

\section{USB Device Hierarchy}

The structure of a USB device is hierarchal, instead of topological.  The general
structure of a device is defined by the USB specification in Chapter 5.
This structure is made up of different components as defined in the specification,
and any specific device can organize those components as the device requires,
within certain limitations which are also defined in the USB specification.
All devices must have one or more configurations.  As the name suggests,
each configuration represents a different configuration of the device.
Only one of the configurations may be active at a time.  Each of those configurations
must contain one or more interfaces.  An interface represents a certain function
of the device (or configuration).  Any interface has at least one setting,
and it may have alternate settings.  Only one setting may be active at a time.
Each setting may contain zero or more endpoints, and each of those endpoints
contains a pipe.  The device's pipes, which are used to communicate with the
device, are the lowest level of the hierarchy.

\myfigure{figs/logical_device_hierarchy}{Logical Device Hierarchy}{device_hierarchy}

The javax.usb API again closely matches this logical hierarchy.  Figure \ref{device_hierarchy}
shows the logical arrangement of an example device in javax.usb terms.  A physical
device is represented by an instance of the \myinterface{UsbDevice} interface.  In this example,
there are two \myinterface{UsbConfigurations}, which represent the device's configurations,
inside the \myinterface{UsbDevice}.  Those \myinterface{UsbConfiguration}s
contain \myinterface{UsbInterface}s; the \myinterface{UsbInterface}s
contain \myinterface{UsbEndpoint}s; and each \myinterface{UsbEndpoint}
contains a \myinterface{UsbPipe}.  Each of those represents a logical component
of the example device, i.e. the device's configurations, interfaces, endpoints,
and pipes.  To show how alternate settings are handled, inside \myinterface{UsbConfiguration}
number 2, the \myinterface{UsbInterface} number 1 contains two settings, setting 0 and setting 1.

\mysectionend

%%%%%%%%%%%%%%%%%%%%%%%%%%%%%%%%%%%%%%%%
% UsbDevice
%

\section{UsbDevice}

The \myinterface{UsbDevice} interface is the connection between the USB bus's
topology and the device's logical hierarchy.  In addition to
navigational methods, the \myinterface{UsbDevice} interface also has methods
used for identification and communication.  Figure \ref{UsbDevice}
shows most of the methods provided by the \myinterface{UsbDevice} interface.

\myfigure{figs/UsbDevice}{UsbDevice}{UsbDevice}

The navigational methods connect to the \myinterface{UsbPort} above this
device in the bus topology, as well as all the \myinterface{UsbConfiguration}s
below this device in its device hierarchy.  The \myinterface{UsbConfiguration}s
are available by specific configuration number or in a list of
all configurations.

The identification methods provide the three strings defined in
the USB specification, the Manufacturer, Product, and Serial Number.
Also there is a method to get any of the device's string descriptors,
by number.  Finally there is a method to get the device's descriptor.
The descriptor, which is represented by an instance of \myinterface{UsbDeviceDescriptor},
contains methods to get all the values of a device descriptor as
defined in the USB specification.

The communication methods provide access to the device's default control
pipe.  The default control pipe requires \myinterface{UsbControlIrp}s, which contain
meta-information besides the actual data buffer.  The \myinterface{UsbControlIrp}s
may be submitted synchronously or asyncronously.  The submission methods
behave the same, except the synchronous submission method blocks until
the submission is complete, essentially meaning all data has been transferred.
The asynchronous submission method does not block, but returns as soon as
possible, and the submission is completed in a seperate \myclass{Thread}.  There
are also methods that allow submission of a \myinterface{List} of \myinterface{UsbControlIrp}s.

Finally, there are methods that allow adding and removing a listener for
\myclass{UsbDeviceEvent}s.  After adding a \myinterface{UsbDeviceListener}, that listener will
get notified whenever a device event occurs.  Device events will get
fired whenever and data is transferred, or if any error occurs while
transferring data, or if the device is physically disconnected.

\mysectionend

%%%%%%%%%%%%%%%%%%%%%%%%%%%%%%%%%%%%%%%%
% UsbConfiguration

\section{UsbConfiguration}

A \myinterface{UsbConfiguration} is the next interface in the device hierarchy.
As shown in figure \ref{UsbConfiguration}, the \myinterface{UsbConfiguration}
is much less complicated than the \myinterface{UsbDevice}.  It provides some
identification methods and navigational methods, as well as a
method for checking if this configuration is active or not.

\myfigure{figs/UsbConfiguration}{UsbConfiguration}{UsbConfiguration}

The navigational methods are similar to the \myinterface{UsbDevice}'s navigational
methods, they allow access to the interfaces above and below in the
device hierarchy.  There is a method that leads to the \myinterface{UsbDevice}
instance above this in the device hierarchy.  Also there are methods
that lead to the \myinterface{UsbInterface}s below this configuration in the device
hierarchy.  The interfaces are available by number or in a list of all
available interfaces.  If any interface has more than one alternate
setting, and the configuration itself is active, then the active
alternate setting of the interface is provided.  If the configuration
itself is not active, an implementation-dependent alternate setting
is provided.

The identification methods include a method to get the string
descriptor for this configuration, if one exists, and a method to get
the configuration descriptor associated with this configuration.  The
\myinterface{UsbConfigurationDescriptor} provides methods corresponding to the fields
listed in the USB specification.

\mysectionend

%%%%%%%%%%%%%%%%%%%%%%%%%%%%%%%%%%%%%%%%
% UsbInterface
%

\section{UsbInterface}

Next in the device hierarchy is the \myinterface{UsbInterface}.  The \myinterface{UsbInterface} is
similar to those interfaces above it, as shown in figure \ref{UsbInterface};
it has identification, navigation, and status methods.  The navigational
methods are more complicated however, because the interface may have
alternate settings.  Additionally, the status methods are more complicated.

\myfigure{figs/UsbInterface}{UsbInterface}{UsbInterface}

The navigational methods allow access to the \myinterface{UsbConfiguration} above this
interface in the device hierarchy, as well as accessing all the
endpoints below this interface in the device hierarchy.  As usual, the
\myinterface{UsbEndpoint}s are available by number, more specifically endpoint address,
as well as a list of all available endpoints.  In addition to those
navigational methods, there are methods that allow access to any alternate
settings the interface may have.

The identification methods provide the string descriptor for this interface,
if one exists, as well as the interface descriptor associated with this interface.
The \myinterface{UsbInterfaceDescriptor} contains methods corresponding to all the fields
listed in the USB specification.

The status methods are slightly more complicated than other status methods.
There is a method to check if this interface setting is active.  This
method is only true if the parent \myinterface{UsbConfiguration} is active as well as
this \myinterface{UsbInterface} setting.  Also, there are methods to allow claiming
and releasing of the interface, and a method to check if the interface
has been claimed.  The claiming is used to aquire an exclusive lock
on the interface and all the endpoints and their pipes under the interface.
Before sending or receiving data to or from any of the pipes under
this interface, the interface must be claimed.

\mysectionend

%%%%%%%%%%%%%%%%%%%%%%%%%%%%%%%%%%%%%%%%
% UsbEndpoint
%

\section{UsbEndpoint}

The \myinterface{UsbEndpoint} is quite simple in comparison to other device hierarchy
interfaces.  As shown in figure \ref{UsbEndpoint}, the only methods are
navigational methods to access the \myinterface{UsbInterface} above this, and the
\myinterface{UsbPipe} below.  Also, there is a method to access the \myinterface{UsbEndpointDescriptor},
and methods to get the direction and type of endpoint and its associated
pipe.

\myfigure{figs/UsbEndpoint}{UsbEndpoint}{UsbEndpoint}

\mysectionend

%%%%%%%%%%%%%%%%%%%%%%%%%%%%%%%%%%%%%%%%
% UsbPipe
%

\section{UsbPipe}

The \myinterface{UsbPipe} is the lowest interface in the device hierarchy.  As shown in
figure \ref{UsbPipe}, it has navigational and status methods, but most of
its methods are communication methods.  The navigation method allows
access to the \myinterface{UsbEndpoint} that this pipe is associated with, and there
is a method to determine if this pipe is active or not.  A pipe is
active if its parent interface setting is active.

In addition to the method to determine if the pipe is active, there
are methods that allow opening and closing of the pipe, and a method
to determine if the pipe is open.  The pipe must be opened before
it can be used for communication.  Also, the parent interface must
be claimed before the pipe can be opened.

The communication methods are similar to the communication methods
from the \myinterface{UsbDevice} interface, which allow communication on the default
control pipe.  However, this pipe is not neccesarily a control-type pipe,
and so other objects may be used for the communication.  There are
synchronous and asyncronous methods, which behave the same as the
methods from the \myinterface{UsbDevice} interface; the synchronous methods block
until complete, while the asynchronous methods return immediately
and perform processing in a background thread.  If the pipe is a
control-type pipe, the object used for communication must be
a \myinterface{UsbControlIrp}, just as is used with the default control pipe.
However for non-control-type pipes, a \myinterface{UsbIrp} can be used as
well as a simple byte[].  The byte[] is the simplest method
of communication.  It simply transfers all the contained data
to the device, if the pipe direction is host-to-device, or fills up
part or all of the buffer with data from the device, if the pipe
direction is device-to-host.  The \myinterface{UsbIrp} is similar, it contains
a byte[] data buffer also, but there are other methods to limit
the amount of transferred data, or start the transfer at a certain offset
into the buffer, or cause an error if any short packets are detected.

The \myinterface{UsbPipe} also has methods that allow a \myinterface{UsbPipeListener} to be added or removed
from the pipe.  After a \myinterface{UsbPipeListener} has been added, it will receive a
\myclass{UsbPipeEvent} when any data is transferred to or from the device on this pipe,
or when any error occurs while transferring data.  After removing the listener,
it will not receive any more events.

\myfigure{figs/UsbPipe}{UsbPipe}{UsbPipe}

\mysectionend

%%%%%%%%%%%%%%%%%%%%%%%%%%%%%%%%%%%%%%%%
% UsbControlIrp and UsbIrp
%

\section{UsbControlIrp and UsbIrp}

The interfaces used to actually communicate with the device are shown in
figure \ref{UsbIrps}.  The \myinterface{UsbControlIrp} interface is actually an extension
of the \myinterface{UsbIrp} interface, adding methods that represent fields in the special
control-type header required for control communication.  Otherwise, the
\myinterface{UsbControlIrp} is identical to a normal \myinterface{UsbIrp}.  A \myinterface{UsbIrp} has methods that
can be divided into four different categories; the data buffer, status,
short packets, and the \myclass{UsbException}.

\myfigure{figs/UsbIrps}{UsbControlIrp and UsbIrp}{UsbIrps}

First, the data buffer is simply a byte[] but there are additional methods that
are used to modify the transfer of data.  There are methods to set and get the
offset, which indicates what offset into the data buffer the implementation
should use when transferring data.  If the offset is zero, data will be transferred
starting at the beginning of the byte[], if the offset is above zero, data starting
at the offset into the byte[] will be used when communicating with the device.
There are also methods to set and get the length of data to transfer with the device.
Lastly, there is a method to get the actual amount of data transferred with the device.
For an input-direction pipe, this value may be less than the length of data that was
requested from the device; for an output-direction pipe, this should be the full
amount of data to transfer.

The status methods deal with the complete status of the \myinterface{UsbIrp}.  There is a method to
check if the \myinterface{UsbIrp} is complete or not.  There are also methods to wait until the \myinterface{UsbIrp}
is complete; these will block until the \myinterface{UsbIrp} changes status to complete.  There
is a method used to set the \myinterface{UsbIrp} complete or not complete, and finally there is a
method that is used by the implementation to set the \myinterface{UsbIrp} as complete after finishing
processing.  This method will also call the method to set the status of the \myinterface{UsbIrp} as
complete, and cause any \myclass{Thread}s waiting for the \myinterface{UsbIrp} to complete to return from
the methods they are blocking in.

The methods related to short packets are used to set and get the policy that should be
used when handling short packets.  The policy can be set to either accept or reject
short packets.  Short packets will happen if the device transfers less data than the
host was expecting.  Normally this will happen only if the pipe direction is
device-to-host, and the data buffer provided is larger than the amount of data that
the device has to send to the host during a specific communication.  If short packets
are accepted, and a short packet occurs, the communication will complete successfully
and the actual length of transferred data will be less than the size of the provided
data buffer.  If short packets are not accepted, and a short packet occurs, the
\myinterface{UsbIrp} will complete with an error.

Finally, there are methods used to manage any UsbException that may occur during
communication on the pipe.  There is a method to check if an \myclass{UsbException} has
occurred during communication on the pipe, as well as methods to get and set any
\myclass{UsbException} that occurs.

It is important to note that submissions are the only way to communicate
with a device, and although listeners will receive events for all data
transferred on a pipe, that data must be provided via submissions.
Each pipe is unidirectional, and data may flow only one direction on it.
If the pipe is an output pipe, the data located in the byte[] is sent
to the device during submission; however if the pipe is an input pipe,
the byte[] is filled up with data received from the device.  Thus,
if input is expected on an input pipe, one or more data buffers
must be submitted, and only then will data be received from the pipe.
Once all submissions for a pipe are done, no more data will be received on
that pipe until more data buffers are submitted.  If a constant flow
of data is desired, multiple buffers should be submitted, and as each
submission finishes, more buffers should be submitted.  Using only
a single buffer may result in undesirable delays, since the device
may be able to produce data at a faster rate than each submission takes,
especially if there are sudden bursts of data.

\mysectionend

% Add pagebreak
\pagebreak

%%%%%%%%%%%%%%%%%%%%%%%%%%%%%%%%%%%%%%%%
% DCP
%

\section{Default Control Pipe}

The Default Control Pipe is used for much of the communication with a device.
It is a special pipe that is required to be present on all devices, and it is
always available for communication.  It is used for all the configuration-type
communication, as well as communication to get identification information from
the device.

In figure \ref{DCP}, an example of communication using the Default Control Pipe
is shown.  First, the user has to obtain an instance of a \myinterface{UsbControlIrp}.  The
user can use their own implementation of \myinterface{UsbControlIrp}, or they can use the
implemenation provided by the \myinterface{UsbDevice} implementation, as shown in the figure.
In the example, the \myinterface{UsbControlIrp} is created by the \myinterface{UsbDevice}, and the extra
control-type methods are set up during creation.  Next, it should be given
a data buffer, containing data to transfer to the device or a buffer to
fill with data from the device, depending on the direction of communication.
This is the minimum amount of configuration of the \myinterface{UsbControlIrp} required.

To actually perform the communication on the Default Control Pipe, the
\myinterface{UsbControlIrp} is submitted using, in this example, the \mymethod{syncSubmit}
method.  The implementation of this method will pass the buffer down to the
platform's low-level USB subsystem, which will perform the actual transfer of
data.  After the data transfer is complete, the \myinterface{UsbDevice} implementation will
set either the actual length of transferred data or the \myclass{UsbException} on the
\myinterface{UsbControlIrp}, and then \mymethod{complete} the \myinterface{UsbControlIrp}.  This is the
simplest communication example, which does not show the use of data offset or
length, and assumes no listeners are being used.

\myfigure{figs/DCP}{Default Control Pipe}{DCP}

\mysectionend

%%%%%%%%%%%%%%%%%%%%%%%%%%%%%%%%%%%%%%%%
% UsbPipe submission, synchronous
%

\section{UsbPipe submission, synchronous}

For most communication on a device, the Default Control Pipe is not used.
Instead, normal unidirectional pipes are used to transfer data.  Figure
\ref{UsbPipeSync} shows an example of communication using one of the normal
\myinterface{UsbPipe}s.  There are more steps to take before actually submitting the \myinterface{UsbIrp}
than when using the Default Control Pipe.

First, before using any of the \myinterface{UsbPipe}s contained in a \myinterface{UsbInterface} setting,
the \myinterface{UsbInterface} must be claimed.  Then, the \myinterface{UsbPipe} must be opened.  If either
of these steps fails, no communication will be possible on that pipe; the
problem that prevented the claim or the open must be corrected first.  If the
claim and the open succeed, the pipe is ready for communication.  The interface
setting may be left claimed and the pipe may be left open as long as the
interface and its pipes are being used.

To actually communicate on the pipe, data buffers must be provided to the pipe.
In this example, a \myinterface{UsbIrp} is used.  The \myinterface{UsbIrp} implementation may be provided
by the caller, but the example shows using the \myinterface{UsbPipe} to create a \myinterface{UsbIrp}
instance.  After creating the \myinterface{UsbIrp}, at least the data must be set.  Additionally,
the other fields of the \myinterface{UsbIrp} may be set, including the offset, length, and short
packet policy of the \myinterface{UsbIrp}.  Those fields are optional, and their defaults should
be acceptable for the majority of communication.  Once the data buffer and any
optional fields are set, the \myinterface{UsbIrp} must be provided to the \myinterface{UsbPipe}; in the example,
the \mymethod{syncSubmit} method is used.  Similar to the Default Control Pipe
example, the implementation passes the data buffer to the platform USB subsystem,
which performs the actual communication with the device.  After the communication
is complete, the implementation either sets the actual length of the data
transferred, or it sets the appropriate \myclass{UsbException} on the \myinterface{UsbIrp}.  It then
calls the \myinterface{UsbIrp}'s \mymethod{complete} method to indicate the \myinterface{UsbIrp} is completed.
While it is not shown in the example, if any \myinterface{UsbPipeListener}s had been added to
the \myinterface{UsbPipe}, they would receive either a \myclass{UsbPipeDataEvent} or a \myclass{UsbPipeErrorEvent},
depending on whether the communication succeeded or not.

\myfigure{figs/UsbPipeSync}{UsbPipe synchronous submission}{UsbPipeSync}

%%%%%%%%%%%%%%%%%%%%%%%%%%%%%%%%%%%%%%%%
% UsbPipe submission, asynchronous

\section{UsbPipe submission, asynchronous}

Asynchronous submission on a pipe is very similar to synchronous submission.
As shown in figure \ref{UsbPipeAsync}, the initial setup is the same; the
\myinterface{UsbInterface} setting must be claimed, and the \myinterface{UsbPipe} must be opened.  The
\myinterface{UsbIrp} should be created in the same way, and the data buffer set as well
as any of the optional fields such as offset or length.  This example also
shows how to use a \myinterface{UsbPipeListener}.  Before submitting the \myinterface{UsbIrp}, the
listener should be added to the \myinterface{UsbPipe}.  The listener only needs to be added
once to the pipe.  Next, the \myinterface{UsbIrp} is submitted to the pipe using the
\mymethod{asyncSubmit} method.  The submitting \myclass{Thread} should return
immediately, while the \myinterface{UsbIrp} is processed in the background by the implementation.
While the implementation is processing the \myinterface{UsbIrp}, the calling \myclass{Thread} can perform
other actions.  Eventually, it will want to know if the \myinterface{UsbIrp} is complete.  The
calling \myclass{Thread} can then call the \mymethod{waitUntilComplete} which will block
until the \myinterface{UsbIrp} is complete.  

Once the platform USB subsystem has finished transferring the data, it will return
it to the implementation.  The implementation handles it normally, by setting
either the actual length of data transferred, or the \myclass{UsbException} that occurred,
and finally completing the \myinterface{UsbIrp}.  Next the \myinterface{UsbPipeListener} will receive a
\myclass{UsbPipeDataEvent} or \myclass{UsbPipeErrorEvent}, depending on whether the submission was
successful or not.

\myfigure{figs/UsbPipeAsync}{UsbPipe asynchronous submission}{UsbPipeAsync}

\mysectionend

%%%%%%%%%%%%%%%%%%%%%%%%%%%%%%%%%%%%%%%%
% UsbPipe submission, byte[]
%

\section{UsbPipe submission, byte[]}

Submissions using \myinterface{UsbIrp}s are the most flexible type of communication.
Submissions using byte arrays are much more simple, however.  In
figure \ref{UsbPipeByteArray}, both synchronous and asynchronous submission
using a byte[] are shown.  The setup steps, such as claiming the interface
and opening the pipe, are the same.  After these initial steps, in the
first example using asynchronous submission, the byte[] is submitted to
the \myinterface{UsbPipe} using the \mymethod{asyncSubmit} method.  The implementation then
creates a \myinterface{UsbIrp} to represent the provided data buffer, and this \myinterface{UsbIrp} is
returned to the caller.  Then processing continues the same as asynchronous
submission of a \myinterface{UsbIrp}.  In the synchronous example, the data buffer is passed
to the \myinterface{UsbPipe} using the \mymethod{syncSubmit} method, which blocks while the
implementation handles transferring the data.  After the data has finished
transferring, the implementation either returns the actual length of data
transferred, or it throws a \myclass{UsbException} representing the error that
occurred during data transfer.  In both cases, after the data transfer is complete
or an error occurs, any listeners that have been added to the pipe will receive
a \myclass{UsbPipeDataEvent} or \myclass{UsbPipeErrorEvent}.

\myfigure{figs/UsbPipeByteArray}{UsbPipe byte array submissions}{UsbPipeByteArray}

\mysectionend

%%%%%%%%%%%%%%%%%%%%%%%%%%%%%%%%%%%%%%%%
% Hotplugging
%

\section{Hotplugging}

An important feature of the USB bus is the ability for USB devices to
be connected and disconnected while the system is running.  This is called
hotplugging, and system software must be able to handle hotplugging a device.
Hotplugging is fully supported by the javax.usb subsystem.  Figure
\ref{Hotplugging} shows how the user can get notified of hotplugging events.
First, the user can add a \myinterface{UsbServicesListener} to the \myinterface{UsbServices} instance.
When a physical device is connected to the system, the \myinterface{UsbServicesListener}
receives a \myclass{UsbServicesEvent} indicating that a device has been connected to the
system.  If the user is interested in this newly connected device, they can
add a \myinterface{UsbDeviceListener} to the \myinterface{UsbDevice}.  Then, when the device is
disconnected, both the \myinterface{UsbServicesListener} and the \myinterface{UsbDeviceListener} will be
notified with a \myclass{UsbServicesEvent} and a \myclass{UsbDeviceEvent}, respectively.

\myfigure{figs/Hotplugging}{Hotplugging}{Hotplugging}

\mysectionend

% add pagebreak
\pagebreak

%%%%%%%%%%%%%%%%%%%%%%%%%%%%%%%%%%%%%%%%
% Exceptions
%

\section{Exceptions}

The javax.usb subsystem uses many \myclass{Exception}s to represent some of the
errors that can occur on the USB bus.  Figure \ref{Exceptions} shows the
\myclass{Exception}s used.  The base \myclass{Exception} is \myclass{UsbException}, which is extended by
many more specific \myclass{Exception}s such as \myclass{UsbBabbleException}, \myclass{UsbCRCException},
and others.  Also, there are some \myclass{RuntimeException}s such as \myclass{UsbNotActiveException}
and \myclass{UsbNotClaimedException}.

\myfigure{figs/Exceptions}{Exceptions}{Exceptions}

\mysectionend


%%%%%%%%%%%%%%%%%%%%%%%%%%%%%%%%%%%%%%%%
% Security
%

\section{Security}

Security is not yet fully addressed.

\mysectionend

%%%%%%%%%%%%%%%%%%%%%%%%%%%%%%%%%%%%%%%%
% Utilities
%

\section{Utilities}

The API contains several utility classes, all located in the \mypackage{javax.usb.util}
package, which make the API itself easier to use.

\subsection{DefaultUsbIrp and DefaultUsbControlIrp}

There is a default implementation of the \myinterface{UsbIrp} interface, as well as a default
implementation of the \myinterface{UsbControlIrp} interface.  These may be used to create irps
(and control-type irps) to use in submissions.  Additionally, the \myinterface{UsbPipe} itself
contains a method to create \myinterface{UsbIrp} and \myinterface{UsbControlIrp} objects which
the implementation may prefer, and may require less overhead to process than either the default
implementations or any other implementation.  The \myinterface{UsbDevice} also allows creation of
\myinterface{UsbControlIrp}s.  However, any implementation must be accepted by the implementation;
it may not restrict the \myinterface{UsbIrp} implementation nor the \myinterface{UsbControlIrp}
implementation.

\subsection{StandardRequest}

The \myclass{StandardRequest} class provides a way to easily perform standard device requests.
It contains methods which correspond to all standard device requests defined in the USB
specification; however not all those requests may be possible, since some are intended for
use only by the low-level USB subsystem driver(s).

\subsection{Version}

To determine what version of the API is in use, a \myclass{Version} class is
provided in the base \mypackage{javax.usb} package.  It contains methods to determine
the version of the API itself, as well as the version of the USB specification
that the API supports.  It also contains a \mymethod{main} method so it can
be called directly; this method simply prints out the version numbers.

\mysectionend

\end{document}